% !TEX TS-program = pdflatex
% !TEX encoding = UTF-8 Unicode

% This is a simple template for a LaTeX document using the "article" class.
% See "book", "report", "letter" for other types of document.

\documentclass[11pt]{article} % use larger type; default would be 10pt

\usepackage[utf8]{inputenc} % set input encoding (not needed with XeLaTeX)

%%% Examples of Article customizations
% These packages are optional, depending whether you want the features they provide.
% See the LaTeX Companion or other references for full information.

%%% PAGE DIMENSIONS
\usepackage{geometry} % to change the page dimensions
\geometry{a4paper} % or letterpaper (US) or a5paper or....
\geometry{margin=.8in} % for example, change the margins to 2 inches all round
% \geometry{landscape} % set up the page for landscape
%   read geometry.pdf for detailed page layout information

\usepackage{graphicx} % support the \includegraphics command and options
\graphicspath{ {Ph020_Labs/} }

% \usepackage[parfill]{parskip} % Activate to begin paragraphs with an empty line rather than an indent

%%% PACKAGES
\usepackage{booktabs} % for much better looking tables
\usepackage{array} % for better arrays (eg matrices) in maths
\usepackage{paralist} % very flexible & customisable lists (eg. enumerate/itemize, etc.)
\usepackage{verbatim} % adds environment for commenting out blocks of text & for better verbatim
\usepackage{subfig} % make it possible to include more than one captioned figure/table in a single float
% These packages are all incorporated in the memoir class to one degree or another...

%%% HEADERS & FOOTERS
\usepackage{fancyhdr} % This should be set AFTER setting up the page geometry
\pagestyle{fancy} % options: empty , plain , fancy
\renewcommand{\headrulewidth}{0pt} % customise the layout...
\lhead{}\chead{}\rhead{}
\lfoot{}\cfoot{\thepage}\rfoot{}

%%% SECTION TITLE APPEARANCE
\usepackage{sectsty}
\allsectionsfont{\sffamily\mdseries\upshape} % (See the fntguide.pdf for font help)
% (This matches ConTeXt defaults)

%%% ToC (table of contents) APPEARANCE
\usepackage[nottoc,notlof,notlot]{tocbibind} % Put the bibliography in the ToC
\usepackage[titles,subfigure]{tocloft} % Alter the style of the Table of Contents
\renewcommand{\cftsecfont}{\rmfamily\mdseries\upshape}
\renewcommand{\cftsecpagefont}{\rmfamily\mdseries\upshape} % No bold!

%%% END Article customizations

%%% The "real" document content comes below...

\title{Ph020 Lab 3-  Numerical Solutions to Differential Equations}
\author{Arjun Goswami}
%\date{} % Activate to display a given date or no date (if empty),
         % otherwise the current date is printed 

\begin{document}
\maketitle

\section{Motion of Mass on a Spring}
\includegraphics[width=15cm]{explicit_spring_motion}
\\
This was  implemented with the explicit Euler method. The black sinusoid represents the x-position as a function of time and the blue sinusoid represents the velocity as a function of time.

\section{Global Error Time Evolution of Explicit Euler Method}
\includegraphics[width=10cm]{explicit_spring_error}
\\
The black curve represents the error in x-position as a function of time and the blue curve represents the error in velocity as a function of time.

\section{Truncation Error vs step size $h$}
\includegraphics[width=10cm]{truncation_error}
\\
For values of $h_0=.0001,\frac{h_0}{2}, \frac{h_0}{4}, \frac{h_0}{8}, \frac{h_0}{16}$

\section{Energy Evolution of the Explicit Euler Method}
\includegraphics[width=10cm]{energy_evolution_explicit}
\\
The energy in the system should be conserved. This holds for the analytical solution, whose energy as a function of time is the constant blue line. However, the numerical solution's energy (black) is not conserved, and instead grows with time.

\section{Global Error Evolution of the Implicit Euler Method}
\includegraphics[width=10cm]{implicit_global_error}
\\
The global error evolution of the implicit Euler method and that of the explicit method are virtually identical, and are given by (nearly) the same curve, plotted above. 

\section{Energy Evolution of the Implicit and Explicit Euler Methods}
\includegraphics[width=10cm]{energy_evolution_implicit_explicit}
\\
The energy evolution of the implicit Euler method is given by the black curve, and that of the explicit Euler method is given by the blue curve. While the energy with the explicit method increases with time, the energy of the implicit method decreases with time. 

\section{Phase Space Geometry of Explicit Euler Method}
\includegraphics[width=10cm]{explicit_phasespace}
\\
The phase space geometry is simply the x-v plane. Deviations from a closed circle are apparent due to the numerical error of the explicit Euler method. 

\section{Phase Space Geometry of Implicit Euler Method}
\includegraphics[width=10cm]{implicit_phasespace}

\section{Phase Space Geometry of Symplectic Euler Method}
\includegraphics[width=10cm]{symplectic_phasespace}
\\
The symplectic Euler method does not have any apparent deviations from a closed circle in its phase-space trajectory.

\section{Energy Evolution of the Symplectic Euler Method}
\includegraphics[width=10cm]{symplectic_energy_evolution}
\\
The time-evolution of the energy of the system as calculated by the symplectic Euler method is sinusoidal.


\end{document}
